\documentclass[runningheads]{llncs}
\usepackage[T1]{fontenc}
\usepackage{graphicx}
\usepackage{color}
%\renewcommand\UrlFont{\color{blue}\rmfamily}
%\urlstyle{rm}
%
\begin{document}
%
\title{Conditional Variational Autoencoders with Fuzzy Inference}
\titlerunning{CVAE Fuzzy Inference}
%
\author{Yury Gurov\inst{1}\orcidID{0000-0002-7033-9996} \and
Danil Khilkov\inst{1}\orcidID{0000-0001-9284-6924}}
\authorrunning{Y. Gurov and D. Khilkov}
%
\institute{NIIAS Institute of Informatization, Automation and Communication in Railway Transport, Russia, Moscow
109029 Nizhegorodskaya str., 27 bldg. 1
\email{info@vniias.ru}\\
\url{www.vniias.ru/} 
}
%
\maketitle              % typeset the header of the contribution
%
\begin{abstract}
We present an approach to constructing Conditional Variational Autoencoders (C-VAE) models with fuzzy inference during classification.
This preserves disentangling capabilities of VAE and at the same time performs latent space clusterization.
Fuzzy C-VAE model provides useful features for anomaly detection, utilizing partially labeled datasets and controlled generation of new samples.
%
\keywords{fuzzy logic  \and deep learning \and fuzzy inference \and Conditional Variational Autoencoders \and fuzzy cvae \and neuro-fuzzy}
\end{abstract}
%
%
%
\section{Introduction}

Идей гибридных нейро-нечетких систем существует достаточно большое количество \cite{DECAMPOSSOUZA2020106275}.
Особым интересом пользуются системы, которые позволяют получить одновременные преимущества глубоких нейросетевых моделей и human-like reasoning.
В этой работе мы фокусируемся на задаче повышения интерпретируемости выводов внутренних слоев вариационных аутоэенкодеров \cite{kingma2022autoencoding,Kingma_2019}.
Продолжением развития идей VAE явились conditional Variational Autoencoders \cite{debbagh2023learning}, в которых процесс моделирования выходного многообразия VAE управляется неким набором управляющих параметров, интерепретируемых как экспертное знание.


\subsubsection{Contributions} In this work, we propose an fuzzy layer representation which can be used to learn conditional VAE models.


The source code available in GitHub repository (https://github.com/kenoma/pytorch-fuzzy)

\section{Related work}
В этом направлении предпринимались определенные попытки применить нечеткую кластеризацию к латентому слою \cite{Bolat2020}.
An amortised variational distribution for learning the missing auxiliary covariates and a method that maximises the evidence lower bound objective while simultaneously marginalising uncertainty associated with the missing covariates was introduced at \cite{RAMCHANDRAN2024110113}. 

\section{Methods}

\subsection{Problem setting}

\subsection{Variational Autoencoders}

\subsection{Conditional VAE}

\subsection{Fuzzy inference}

\subsection{Learning Fuzzy C-VAE}


\section{Experiments}

\subsection{Latent space clusterization}

\subsection{Learning on partially labeled dataset}

\subsection{Anomaly detection}

\section{Discussion}

In this paper, we introduced a novel fuzzy inference layer to improve the performance of conditional VAEs. 
We achieve this by making trainable multidimentional representation of fuzzy term.
The method that we proposed is applicable to a variety of conditional VAE models. 
The efficacy of our proposed method was demonstrated on MNIST dataset. 

\bibliographystyle{splncs04}
\bibliography{refs}

\end{document}
